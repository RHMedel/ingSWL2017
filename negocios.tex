
\chapter{Modelos de negocio con software libre}




\section{Financiación de los Proyectos de Software Libre}

La financiación externa puede considerarse como un tipo de patrocinio, aunque el mismo no tiene porqué ser desinteresado.

\subsection{Financiación Pública}

La entidad financiadora puede ser el gobierno o una institución pública. Suele ser similar a Investigación y Desarrollo (I+D). Usualmente no busca recuperar la inversión, pero tiene objetivos claros (favorecer la industria, promover cierta tecnología…)

{\bf Científica: } es la más común. Interés por no comercializar la herramienta o incluso que otros utilicen la herramienta distribuida como SW libre.

{\bf Promoción de estándares: } a través de una implementación de referencia. Por ejemplo, muchos de los protocolos de Internet. Quien se encarga de la promoción usualmente es una entidad pública.

{\bf Social: } para ayudar al acceso universal de la sociedad a la información. 

Un notorio caso de financiación pública para el desarrollo de OSS fue el compilador de Ada, GNAT, financiado por el Departamento de Defensa de EEUU, ya que la primera versión de Ada (Ada83) no tuvo la adopción esperada debido a la tardía disposición de un compilador del lenguaje.


\subsection{Financiación Privada sin ánimo de lucro}

La financiación es llevada adelante por fundaciones u organizaciones no gubernamentales. La motivación es producir SW libre para su uso en algún ámbito relevante de acuerdo al criterio de la entidad financiadora o contribuir a resolver un problema. Es similar a la financiación pública, pero marcado por los objetivos de la entidad que financia.
Un ejemplo es la FSF (Free SW Foundation).


\subsection{Financiación por quien necesita mejoras}

Cuando alguien necesita mejoras de un SW libre para su uso interno, funcionalidades especiales o corrección de errores.
Como ejemplo se puede citar Corel y Wine. Corel trasladó sus productos a GNU/Linux para ahorrar en el desarrollo necesario para añadir funcionalidad de Windows a los programas de Corel, contratando a Macadamian que contribuyó con sus mejoras al proyecto Wine. Tanto Corel como Wine salieron beneficiados.

\subsection{Financiación con Beneficios Relacionados}

Normalmente los beneficios no son exclusivos, pero la cuota de mercado es suficiente como para que no se preocupe o tiene una ventaja competitiva clara.

{\bf Libros: } se venden manuales, guías de uso, textos para cursos, relacionados con el programa.

{\bf Hardware: } otros fabricantes de HW pueden también beneficiarse de las mejoras en el SW libre que usa ese HW, pero la ventaja de quien colabora al desarrollo del SW viene por lograr que las mejoras al SW libre se realicen con prioridad

{\bf CD con Programas: } empresas financian ciertos desarrollos que luego aplican a la distribución de SW. En este caso la financiación se hace con ánimo de lucro.
Un ejemplo es el aporte de la editorial O’Reilly al desarrollo de Perl; la editorial es además una de las principales editoras sobre temas relacionados con Perl, aunque O’Reilly no tiene la exclusividad de la edición de ese libro. Red Hat contribuyó con el desarrollo de componentes de GNOME, consiguiendo un entorno de escritorio para su distribución, aunque otros distribuidores también se vieron beneficiados con la contribución de Red Hat, aunque no en la misma medida

\subsection{Financiación como inversión interna}

Algunas empresas como parte de su modelo de negocio desarrollan SW libre. En otros casos el SW libre se puede desarrollar para satisfacer necesidades de la propia empresa, y que después decida liberar

\subsection{Otros modelos de financiación}

{\bf Utilización del mercado para poner en contacto a desarrolladores y clientes: } sobre todo para pequeños desarrollos.

{\bf Venta de bonos para financiar un proyecto: } un desarrollador escribe la especificación de una mejora o de una idea nueva, estima el costo y emite bonos para su construcción. Estos bonos se ejecutan sólo si el proyecto se termina. Cuando el desarrollador ha vendido suficientes bonos comienza el desarrollo, y en ese momento paga sus deudas y se queda con algún beneficio. Quien financia tiene además interés que dicha herramienta le ayude con una necesidad propia y a la vez no tiene que correr con todo el costo de desarrollo.

{\bf Cooperativa de desarrolladores: } en lugar de trabajar para una empresa, los desarrolladores se nuclean en una organización similar a una cooperativa. Por el resto su funcionamiento es similar al de una empresa



\section{Modelo de Negocios Basados en Software Libre}

\subsection{Mejor conocimiento} 

Una empresa trata de rentabilizar su conocimiento de un producto libre. La ventaja competitiva estará de la mano del mayor conocimiento por parte de la empresa sobre el producto: desarrollos a medida, adaptaciones o integraciones.

\subsection{Mejor conocimiento con limitaciones} 

Se trata de evitar que otros compitan, como en el caso anterior, poniendo barreras, como patentes o licencias propietarias que afectan a una parte pequeña pero fundamental del producto desarrollado. Pueden considerarse como modelos mixtos. 

\subsection{Fuente de un producto libre}

Es similar al primero, salvo que la empresa que lo utiliza es la productora. Esta empresa tiene una clara ventaja competitiva al ser quien desarrolla, controla su evolución y disponerlo antes de la competencia. 

\subsection{Fuente de un producto con limitaciones}

Se busca limitar la competencia. Entre las limitaciones más habituales están:

{\bf Distribución propietaria durante un tiempo, luego libre: } primero se ofrece como SW propietario y luego de un tiempo es libre. La empresa obtiene ingresos de los clientes que quieren las nuevas versiones lo antes posible, y la competencia solo puede comercializar las versiones más viejas, libres.

{\bf Distribución limitada durante un tiempo: } El SW es libre desde que se comienza a distribuir, pero el productor en un momento inicial lo distribuye sólo a sus clientes, que le pagan por ello (como contrato de mantenimiento). Luego de un tiempo se distribuye a cualquiera. El costo que debe soportarse es que al estar demorada la contribución no podrá beneficiarse tempranamente de las contribuciones externas.

\subsection{Licencias especiales}

El producto se distribuye bajo dos licencias o más, siendo al menos una de ellas de SW libre, y la/las otra/s propietarias. Suelen complementarse con la oferta de consultoría y desarrollos relacionados con el producto.

\subsection{Venta de marca}

Muchos clientes están dispuestos a pagar el extra por comprar una marca, para establecer una imagen. 
Los casos más conocidos son las empresas que comercializan versiones de GNU/Linux.


\section{Otras Clasificaciones de Modelos de Negocio}

\subsection{Clasificación de Hecker}

Esta clasificación fue la más usada por la publicidad de la OSI:

\begin{description}
	\item[$\cdot$ Suppor seller] venta de servicios relacionados con el producto): la empresa promueve el SW libre y vende servicios como consultoría, adaptación...
	
	\item[$\cdot$ Loss leader] (venta de otros productos propietarios): el SW libre se usa para promover la venta de productos propietarios relacionados con él

	\item[$\cdot$ Widget frosting ] (venta de hardware): el negocio fundamental está en la venta del HW, para lo cual el SW es necesario

	\item[$\cdot$ Accessorizing] (venta de accesorios): se comercializan libros, dispositivos informáticos, etc...

	\item[$\cdot$ Service enabler ] (venta de servicios): el SW libre sirve para crear un servicio (normalmente accesible en línea), del que la empresa obtiene beneficio.

	\item[$\cdot$ Brand licensing] (venta de marca): se registran marcas que se asocian a programas OS. Luego se obtienen beneficios cuando se venden los derechos de uso de las marcas.

	\item[$\cdot$ Sell it, free it ] (vende, libera): primero se ofrece la versión del producto libre. Si tiene éxito, la siguiente se ofrece como SW propietario durante un tiempo, luego del cual se libera, y para ese momento se comienza a distribuir una nueva versión propietaria

	\item[$\cdot$ Software franchaising] (franquicia de software): una empresa franquicia el uso de sus marcas, relacionadas con un programa libre determinado

\end{description}


\section{Impacto sobre las Situaciones de Monopolio}


Cualquier iniciativa dedicada a romper una situación donde un producto domina claramente el mercado estará destinada a producir más de lo mismo: si tiene éxito, vendrá otro a ocupar ese hueco y en breve habrá otro dominante. Solo los cambios tecnológicos producen durante un tiempo la inestabilidad suficiente como para que nadie domine claramente.
El problema no es que haya un producto dominante, sino que haya solamente una empresa que lo ofrezca al mercado. El SW libre ofrece la alternativa a esa situación.
Lo saludable es que cualquiera sea el producto, muchas empresas puedan competir para proporcionarlo, mejorarlo, adaptarlo y ofrecer servicios alrededor de él.

\subsection{Elementos que favorecen a los productos dominantes}

{\bf Formato de datos:  } cuando mucha gente utiliza un formato de dato dado, la presión para usarlo es enorme.

{\bf Cadenas de distribución: } cuando el producto es dominante es fácil su distribución porque todas las cadenas lo buscan para poder ofrecerlo. Si el producto es minoritario le resulta difícil acceder a su comercialización y llegar a los usuarios.

{\bf Marketing: } el boca a boca es una gran propaganda, cuando son muchos los que lo usan. 

{\bf Inversión en formación: } una vez aprendido el uso de una herramienta, se está muy motivado para no cambiar. Además, usualmente esa herramienta es la que domina, por lo que es más fácil encontrar material y personas que ayuden a aprenderla.

{\bf Software preinstalado: } el usuario lo quiere. Además, el SW que viene preinstalado es el más utilizado.

\subsection{El mundo del software propietario}

En este mundo, la aparición de un producto dominante en un segmento cualquiera equivale a un monopolio por parte de la empresa que lo produce.
La empresa tiene gran control sobre el producto, ellos marcan su evolución, su calidad, etc. 
Esta situación hace aparecer los peores efectos económicos del monopolio, entre ellos la falta de motivación de la empresa líder para acercar el producto a las necesidades de sus clientes, ya que están cautivos.

\subsection{La situación del software libre}

Con el SW libre, un producto dominante no se traduce en monopolio de la empresa. Cualquier empresa puede trabajar en él, mejorarlo, adaptarlo a las necesidades del cliente y ayudar a su evolución. Muchos querrán trabajar en esta mejora y si la empresa que lo desarrolló en primera instancia quiere permanecer en el negocio tendrá que competir contra todas ellas y estará muy motivada para hacer evolucionar al producto. Naturalmente tendrá la ventaja de un mejor conocimiento del programa. 
Los usuarios retoman el control.
Como ejemplo de productos libres que son dominantes en su sector se puede nombrar Apache y las distribuciones de GUN/Linux.

\subsection{Estrategias para constituirse en monopolio con software libre}

Son estrategias comunes a otros sectores económicos, pero una es específica del mercado del SW: {\bf la aceptación de productos certificados por terceros.}
Cuando una empresa quiere distribuir SW que funcione en combinación con otras aplicaciones, es común “certificar” la combinación. El fabricante se compromete a ofrecer servicios (actualización, soporte, resolución de problemas) sólo si el cliente está usando el producto en un entorno certificado por él. Si en este segmento la certificación es importante, estamos ante una nueva situación de monopolio.
Las situaciones monopólicas no son fáciles de conseguir, y para conseguirlas se debe utilizar mecanismos “no informáticos”.

\end{document}

